
El sistema plantea una estructura de MVC(Modelo-Vista-Controlador). Para poder comprender esto de mejor manera analizaremos cada uno de sus componentes por separado.

\textbf{Modelo}

\begin{itemize}
    \item \textbf{WeightMeasurementModel}
\end{itemize}

\begin{center}
\begin{lstlisting}
class WeightMeasurementModel {
  int id;
  DateTime date;
  List<double> weights;
  double averageWeight;
  double? difference;
  String? state;
}
\end{lstlisting}
\end{center}

Es la instancia de cada medición de peso, en esta almacenamos los 10 valores de las balanzas, su promedio, la diferencia con el último peso registrado o si es la primera medición con el peso inicial ingresado por el usuario. Por último asignamos un valor al estado para saber si en esa medición subió o bajó de peso.

\begin{itemize}
    \item \textbf{PersonModel}
\end{itemize}

\begin{center}
\begin{lstlisting}
class WeightMeasurementModel {
  int id;
  String name;
  double initialWeight;
  List<WeightMeasurementModel> measurements;
}
\end{lstlisting}
\end{center}

Es la instancia de cada persona que va a pesarse, a esta se le asigna un identificador único, un nombre y una lista de mediciones, de esta manera asociamos las mediciones a cada persona y podremos acceder a ellos para usarlos en la vista.

Adicional se calcula el promedio. métodos de ingreso de personas dependiendo si es su primera medición y sirve para poder extraer los datos y mostrarlos en la ui.

\textbf{Controlador}

\begin{center}
\begin{lstlisting}
class PersonController with ChangeNotifier{
  List<PersonModel> _persons = [];
  int _nextPersonId = 1;
\end{lstlisting}
\end{center}

Es el controlador central, este usa change Notifier para poder recargar la ui (especialmente las listas) después de realizar un ingreso de una nueva persona o medida, este controlador maneja la lista central de personas que es la que manejara las nuevas instancias. En esta clase también tenemos las funciones que controlan inserciones y formatos de visualización.

El controlador actúa como una capa intermedia entre los modelos y la interfaz de usuario, permitiendo que los organismos y páginas se mantengan enfocados únicamente en la presentación, sin preocuparse por la gestión del estado o la persistencia temporal de los datos.

\textbf{Flujo General}

El flujo principal inicia en la página DietPage, que representa la pantalla de inicio del sistema. En ella, el usuario encuentra un formulario para registrar una nueva persona y una lista con las personas ya registradas.

El registro de una persona se realiza a través del PersonFormOrganism, donde el usuario ingresa el nombre y el peso inicial. Tras validar el formulario, se invoca al método addPerson del controlador, que crea y almacena una nueva instancia de PersonModel. Luego, el formulario se limpia y la interfaz se actualiza para reflejar el nuevo registro.

La lista de personas registradas se muestra mediante el PersonListOrganism, donde cada persona aparece dentro de una tarjeta (Card) que incluye su nombre y un distintivo de estado (badge) que indica si su peso ha aumentado o disminuido. Al seleccionar una persona, se navega a la vista de detalle (PersonDetailOrganism), donde se pueden consultar las mediciones históricas y registrar nuevas entradas.
