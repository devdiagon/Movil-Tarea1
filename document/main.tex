\documentclass[12pt,letterpaper]{article}

% =====================
% Configuración general
% =====================
\input{config/libs.tex}
\input{config/format.tex}

% =====================
% Documento
% =====================
\begin{document}
\renewcommand{\figurename}{Ilustración} 
{\textbf{\textcolor{azulOscuro}{INFORME TAREA 1}}}

%===========================================================
%===========================================================
% --- Tabla de Datos ---
%===========================================================
%===========================================================

{\setlength{\parskip}{0pt}
\section{Datos Generales}
}

\begin{tabularx}{\textwidth}{|>{\raggedright\arraybackslash}X|>{\raggedright\arraybackslash}X|}
\hline
Título del Informe: & Creación y ejecución de un proyecto \\
\hline
Autor(a): & Carlos Hernández, Olivier Paspuel, Antonio Revilla, Frederick Tipán\\
\hline
Carrera: & Ingeniería en Software \\
\hline
Asignatura o Proyecto: & Desarrollo de Aplicaciones Móviles \\
\hline
Tutor o Supervisor: & Mgtr. Doris Karina Chicaiza Angamarca\\
\hline
Institución: & Universidad de las Fuerzas Armadas ESPE – Matriz Sangolquí \\
\hline
Fecha de entrega: & 23 de octubre de 2025 \\
\hline
\end{tabularx}


%===========================================================
%===========================================================
% --- Introducción ---
%===========================================================
%===========================================================

\section{Introducción}

El siguiente informe presenta el desarrollo de 5 aplicaciones simples en Flutter, el cual es un framework del lenguaje de programación Dart para el desarrollo de aplicaciones móviles multiplataforma. Dichas aplicaciones permitieron distintas maneras de implementar una solución haciendo uso de las herramientas del framework.

Para el desarrollo de cada uno de los ejercicios propuestos, se utilizó el framework Flutter para generar una aplicación móvil multiplataforma visualmente atractiva. Adicionalmente, se utilizaron los IDEs de Visual Studio Code y Andorid Studio para compilar y ejecutar cada aplicación, junto con un emulador de teléfono correspondiente.

Los 5 ejercicios realizados, permitieron familiarizarse con las herramientas que proporciona Flutter para generar interfaces de usuario en dispositivos móviles. También, cada ejercicio permitió elaborar algoritmos que hacen uso eficiente de estructuras como listas, mapas y los distintos tipos de datos que ofrece el lenguaje de programación Dart.

\newpage

%===========================================================
%===========================================================
% --- índice de Contenidos y Figuras ---
%===========================================================
%===========================================================

\renewcommand{\contentsname}{Índice de Contenidos}
{\setlength{\parskip}{0pt}
\tableofcontents
}

\vspace{1.0cm}

\renewcommand{\listfigurename}{Índice de Ilustraciones}
{\setlength{\cftbeforefigskip}{2pt}
\listoffigures
}

\newpage

%===========================================================
%===========================================================
% --- Objetivos ---
%===========================================================
%===========================================================

\section{Objetivos}
\subsection{Objetivo General}
Elaborar un proyecto donde se puedan ejecutar 5 programas distintos en un entorno móvil con Flutter, las cuales permitan explorar las herramientas que ofrece el framework.

\subsection{Objetivo Específicos}
- Reconocer los principales Widgets de Flutter para elaborar interfaces de usuario que tomen, procesen y presenten datos.

- Hacer uso de estados para el manejo dinámico de la presentación de elementos visuales de la interfaz de usuario.

- Utilizar una estructura de carpetas adecuada para organizar código referente a diseño y lógica de programación.

%===========================================================
%===========================================================
% --- Marco Teórico ---
%===========================================================
%===========================================================

\section{Marco Teórico}
\subsection{Widgets en Flutter}
El componente principal de toda aplicación de Flutter, son los Widgets. Estas piezas de código funcionan como bloques de construcción para crear interfaces de usuario (UI). Los widgets se organizan en base a una jerarquía, donde cada widget se anida dentro de otro widget, es de esta manera se genera una especie de “árbol de widgets”, donde cada widget puede obtener un contexto de su padre. \parencite{FlutterDocs2025}

Comúnmente se suele relacionar a los widgets con componentes estructurales definidos, como textos, botones, labels o demás. Lo cierto, es que los widgets son una estructura más compleja, con la cual es capaz de mostrar contenido, establecer temas, ajustar layouts, pero sobre todo, manejar interacciones con el usuario. \parencite{Gill2025}

Al momento de crear un widget, se tiene que generar una clase que herede las características del componente “statelessWidget” o “statefulWidget”. En cualquiera de los casos se \textbf{hereda atributo “key”:} Es un identificador único que ayuda a actualizar secciones específicas de la UI.

\subsection{Stateless widgets}
Es el tipo de widget más simple, pues se lo utiliza cuando el estado del mismo no debe ser mutable, es decir, sus atributos no cambian durante la compilación, por lo que se tendrá el mismo elemento durante toda la aplicación desde que se crea el widget.

Así, este tipo de widget solo implementa el método “build()” que contiene la estructura del widget, por lo que lo convierte en una solución bastante simple, fácil de predecir y optimizada en rendimiento. \parencite{Tllez2024}


\subsection{Stateful widgets}
Este tipo de widget que es capaz de cambiar luego de haber tenido alguna interacción por parte del usuario u otros medios. Para lograr alcanzar esto, se implementa el método “createState()” que sirve para crear un estado del widget declarado. Este estado se lo alacena en una clase separada que hereda la clase “State”. \parencite{Nwogu2023}

Cuando se almacena el estado del widget, se tiene un ciclo de vida, además del “createState()” con el cual se inicializa el widget por primera vez, se compone de los métodos:

\begin{itemize}
  \item \textbf{initState():} Se llama antes de que sea agregado al árbol de widgets, ocurre la inicialización del widget.
  \item \textbf{build():} Contiene la estructura del widget, considerando que será llamado cada vez que se tenga que volver a construir el widget cuando se cambia de estado.
  \item \textbf{setState():} Actualiza el estado del widget, lo que puede involucrar otras variables internas declaradas dentro del estado, es donde se implementa la lógica que posteriormente se verá reflejada en la UI.
  \item \textbf{dispose():} Este método se llama cuando el widget es eliminado del árbol de widgets, elimina cualquiera de los recursos usados por el widget.
\end{itemize}

\subsection{Tipos de widgets}
Existen varios widgets con diferentes propiedades cada uno. Algunos ofrecen mayores capacidades de personalización, por lo que en conjunto pueden lograr armar una UI coherente. \parencite{Pandya2025}

\textbf{Layout:} Define como los widgets se organizan en la pantalla. (\lstinline{Row}, \lstinline{Column}, \lstinline{Stack}, \lstinline{Expanded}, \lstinline{Container})

\textbf{Estructurales:} Proporciona una estructura básica (genérica) para la presentación de la aplicación. (\lstinline{Scaffold}, \lstinline{AppBar}, \lstinline{Drawer})


\textbf{Interactivos:} Permiten a las personas interactuar con la aplicación. (\lstinline{ElevatedButton}, \lstinline{TextButton}, \lstinline{IconButton}, \lstinline{TextField}, \lstinline{Checkbox}, \lstinline{Switch}, \lstinline{Radio})


\textbf{Widgets espeíficos de plataforma:} Se encuentran los ``Material widgets'' para Android y los ``Cupertino widgets'' para iOS, con la finalidad de tener experiencias más parecidas a como si hubiesen sido hechas nativamente.


\textbf{Estilos:} Ayudan a controlar la apariencia del diseño. (\lstinline{Padding}, \lstinline{Align}, \lstinline{Them}, \lstinline{DecoratedBox})

\subsection{Patrón MVC}

El patrón de arquitectura Modelo-Vista-Controlador, mejor conocido como MVC, es uno de los más utilizados dentro del desarrollo de software, debido a su gran simplicidad que permite organizar un proyecto de tal manera en la que pueda ser escalable y mantenible \parencite{Alvarez2023}. Este patrón se compone de 3 capas:

\textbf{Modelo:} Se encarga de trabajar directamente con los datos que se consumen en una aplicación. Generalmente se accede a una base datos, consumiendo sus recursos y realizando operaciones CRUD. De esta manera proporciona interfaces de programación para acceder a los recursos y poder manipularlos desde otras partes del código.

\textbf{Vista:} Muestra el código referente a la interfaz de usuario y todo lo referente a la usabilidad y experiencia de usuario. Es la parte de la personalización de la aplicación, siendo la encargada de actualizar estados y mostrar datos de manera atractiva en la pantalla. En otros casos también se encarga de manejar las rutas y la navegación entre pantallas, estableciendo optimizaciones necesarias para el mejorar el rendimiento.

\textbf{Controlador:} Su principal funcionalidad es la de servir como un enlace entre la vista y el modelo. Lo más común es que la vista llame al controlador y este controlador llame a alguna operación del modelo para obtener datos. Además, el controlador es el encargado de contener la lógica de negocio, por lo que puede realizar ciertas manipulaciones a los datos traídos por el modelo para llegar a hacer alguna acción en concreto.

\subsection{Atomic Design}
Es una metodología de diseño que parte de la idea simple de como se organizan los seres vivos. Su filosofía se basa en ir creando componentes reutilizables y personalizables para agilizar el desarrollo y favorecer la reutilización de componentes \parencite{Garca2024}. Todo esto siguiendo una estructura de diseño anteriormente preestablecida, como paletas de color, tamaño de texto, fuente y demás decisiones de diseño para aumentar la experiencia de usuario.

Sus fundamentos se basan en las siguientes estructuras:

\textbf{Átomos:} Son unidades básicas de diseño, tratándose de elementos sencillos como iconos, botones, inputs o etiquetas. Se llega a obtener un átomo cuando ser reconoce que no se lo puede descomponen en partes más pequeñas que esta.

\textbf{Moléculas:} Se obtienen al combinar distintos átomos, aunque siguen siendo de nivel elemental, puesto que puede ser reutilizable en varias partes de la aplicación. Un ejemplo podría ser composiciones definidas para una barra de búsqueda.

\textbf{Organismos:} Son componentes más grandes y complejos, puesto que incorporan varias moléculas, generalmente ya se tratan de piezas reconocibles dentro de la interfaz. Ejemplos pueden ser una barra de navegación o una tarjeta de un producto.

\textbf{Plantillas:} En este punto ya se define la estructura de una página o vista. Se encarga de organizar varios elementos a la vez para definir la disposición de los elementos en la pantalla.

\textbf{Páginas:} Esta es el nivel final, donde toma como base una plantilla y luego se la llena con contenido real para así presentarla al usuario.


%===========================================================
%===========================================================
% --- Desarollo ---
%===========================================================
%===========================================================

\section{Desarrollo}

El proyecto se encuentra cargado en el siguiente repositorio de GitHub: 

\href{https://github.com/vieerr/multipurpose-app}{\color{blue}\underline{https://github.com/vieerr/multipurpose-app}}

\subsection{Esctructura del proyecto}
El proyecto se desarrolló bajo el nombre \lstinline{multipurpose-app}, y desde el principio se organizaron las carpetas de manera que cada sección del proyecto quedara clara y ordenada. Esto no solo ayuda a mantener todo estructurado, sino que también hace que implementar los distintos módulos sea mucho más sencillo y ágil.

\begin{figure}[H]
    \centering
    \includegraphics[width=0.4 \textwidth, height=5cm, keepaspectratio]{estructura_proy.png}
    \caption{Estructura de carpetas de proyecto}
    \label{fig:strpry}
\end{figure}

La navegación dentro de la app se maneja desde \lstinline{home.dart} usando una barra inferior, el \lstinline{NavigationBar}, que facilita moverse entre los distintos módulos. Para saber qué pantalla está activa en cada momento, se usa la variable \lstinline{currentPageIndex}. Cada vez que el usuario toca una de las pestañas, el método \lstinline{onDestinationSelected} actualiza ese índice, y gracias a eso el body del \lstinline{Scaffold} se encarga de mostrar dinámicamente el widget que corresponde a la sección seleccionada.

La interfaz también cuenta con una barra superior, o \lstinline{AppBar}, donde se muestra el título de la aplicación junto con algunos accesos rápidos. Al trabajar con un \lstinline{StatefulWidget}, cada módulo puede conservar su propio estado de manera independiente, lo que significa que las transiciones entre secciones son suaves y cada parte de la app funciona de forma autónoma sin interferir con las demás.


\subsection{Ejercicio 11 (POS)}
\textbf{Paso 1.- Estructuración según el patrón MVC:}
Para comenzar con el desarrollo de los ejercicios se decidió en un inicio aplicar el patrón Modelo-Vista-Controlador (MVC) de esta forma se estructuró el código del sistema que se basa en un POS o punto de venta en español, de esta forma se logra asegurar que cada parte o módulo del programa tenga responsabilidades claras y bien definidas. Para el desarrollo de la app con el patrón, se crearon tres carpetas principales dentro del proyecto:
\begin{itemize}
    \item model
    \item view
    \item controller
\end{itemize}

\textbf{Paso 2: Creación de archivos .dart:}
Para el desarrollo de cada componente dentro del patrón MVC implementado, se crearon diferentes archivos donde se desarrollaron diferentes clases para resolver el problema planteado: 

\begin{itemize}
    \item \textbf{Controller:} 
        \begin{itemize}
            \item invoiceController.dart
            \item posController.dart
        \end{itemize}
    \item \textbf{Model:} 
        \begin{itemize}
            \item productModel.dart
            \item invoiceModel.dart
            \item posModel.dart
        \end{itemize}
    \item \textbf{View:} 
        \begin{itemize}
            \item invoiceView.dart
            \item invoiceHistoryView.dart
            \item posView.dart
            \item \textbf{Components (carpeta):}
                \begin{itemize}
                    \item productsDropdown.dart
                \end{itemize}
        \end{itemize}
\end{itemize}

\textbf{Paso 3: Desarrollo de los componentes (modelo $\rightarrow$ controlador $\rightarrow$ vista)}  

\textbf{Modelos:}  
\begin{itemize}
    \item \textbf{productModel.dart}: Dentro de este archivo se colocó la definición de la clase \texttt{Product}, que básicamente es el modelo que va a representar un producto dentro de la app. Dentro de la clase se tienen los atributos: \texttt{id}, \texttt{name} y \texttt{price} que se establecen como valores fijos que no van a cambiar durante la ejecución de la app, mientras que \texttt{quantity} es una variable que sí puede cambiar o modificarse, aunque comienza con un valor inicial establecido por defecto de 1.

    Después se define el constructor en el cual se asegura que al momento de crear un producto, se proporcionen todos los datos esenciales. Además también se crearon los métodos \texttt{increaseQuantity} y \texttt{decreaseQuantity} que de forma resumida lo que hacen es ajustar la cantidad, para evitar que llegue a ser negativa o valores que no tienen sentido. Se puede decir que, esta clase establece la estructura principal para gestionar los productos dentro de la app.
    
    \item \textbf{invoiceModel.dart}: Dentro de este archivo .dart se define la clase \texttt{InvoiceModel}, esta se va a encarga de controlar los productos dentro de una factura. Para lograr esto dentro a nivel de código se importa \texttt{product\_model.dart} y se utiliza la clase \texttt{Product}, algo importante es que aquí se almacenan los elementos en un \texttt{Set<Product>} para de esta forma poder evitar los duplicados.

    Por otra parte dentro del código el constructor se encarga de recibir un conjunto de productos cuando se crea una factura. También el método \texttt{addProduct(Product product)} realiza una verificación de si el producto ya se encuentra en la lista comparando su identificador único o \texttt{id}, si lo encuentra, lo que realiza es que aumenta su cantidad, y si no lo encuentra, lo agrega al conjunto. Por ultimo la función \texttt{decreaseProductQuantity(String productId)} busca el producto por su identificador único y se encarga de disminuir su cantidad, también esta función verifica que si llega a cero el producto se elimine.
    
    \item \textbf{posModel.dart}: En este archivo se define la clase \texttt{PosModel}, que funciona como la base principal del punto de venta dentro de la app. De forma general permite gestionar todas las facturas registradas, en esta se importa \texttt{invoice\_model.dart} y también se utiliza la clase \texttt{InvoiceModel}.

    La clase permite tener una lista de facturas a través de una lista: \texttt{List<InvoiceModel> invoices}, esta permite almacenar diferentes movimiento o transacciones en la memoria. El constructor recibe la lista de facturas al crear la instancia, también el método \texttt{addInvoice(InvoiceModel invoice)} hace que sea más fácil agregar nuevas facturas. En conjunto, toda la clase permite manejar las operaciones de venta y mantener un registro más organizado de todas las facturas generadas en la app.
\end{itemize}

\textbf{Controladores:}  
\begin{itemize}
    \item \textbf{invoiceController.dart}: Dentro del archivo se define la clase \texttt{InvoiceController}, la cual maneja toda la lógica que está relacionada con las facturas. Para lograr esto, la clase importa \texttt{invoice\_model.dart} y \texttt{product\_model.dart}, lo que le permite manejar y manipular los productos y las facturas.

    También la clase mantiene una instancia estática de \texttt{InvoiceModel} que guarda los productos actuales. Su método \texttt{agregarProducto} transforma el precio que se recibe como texto a un valor numérico solo si es válido y crea un \texttt{Product} que se añade a la nueva factura, de lo contrario se genera una excepción. También permite disminuir la cantidad de un producto, obtener la lista de productos que no puede cambiar, generar una copia del modelo de factura y reiniciar los productos.
    
    
    \item \textbf{posController.dart}: En este archivo se define la clase \texttt{PosController}, que trabaja como el controlador del punto de venta y se ocupa de intervenir entre la lógica del sistema y los modelos de datos. Para lograrlo, importa \texttt{invoice\_model.dart} y \texttt{pos\_model.dart}, para tener acceso a las clases \texttt{InvoiceModel} y \texttt{PosModel}.

    La clase conserva una instancia estática de \texttt{PosModel} que se encarga de guardar todas las facturas generadas. Su método \texttt{addInvoice(InvoiceModel invoice)} permite verificar que la factura tenga productos antes de añadirla, pero si está vacía, no se ejecuta. Por otra parte \texttt{previousInvoices} devuelve una lista que no puede modificar facturas previas, evitando que puedan existir modificaciones externas, y \texttt{resetInvoices()} permite vaciar o limpiar todas las facturas que se encuentren almacenadas.
\end{itemize}


\textbf{Vista:}  
\begin{itemize}
    \item \textbf{posView.dart}: Este archivo define la clase \texttt{PosView}, la cual representa la interfaz principal del punto de venta dentro de la app, que está construida con Flutter. Esta vista integra los controladores \texttt{InvoiceController} y \texttt{PosController} para manipular los productos y facturas, permite agregar productos, ver la factura actual y acceder al historial de transacciones.
    
    La interfaz se encuentra estructurada con un \texttt{Scaffold} que integra la vista de la factura y varios botones para diferentes acciones como cerrar caja, agregar productos y generar la factura. Cada botón que genera una interacción actualiza el estado de la app, al facturar, se muestra un mensaje de confirmación para el usuario.

    \item \textbf{invoiceView.dart}:  Este archivo define la clase \texttt{InvoiceView}, la cual permite construir la interfaz de la factura dentro de la app. La clase recibe un \texttt{InvoiceController} que permite manejar la lógica de los productos y un número de factura que se muestra en la interfaz.

    Esta vista da la posibilidad de elegir productos desde un \texttt{ProductsDropdown} y agregarlos a la factura con el método \texttt{\_onProductSelected}, y actualiza la lista en tiempo real. Los productos que se añaden se muestran en un \texttt{ListView}, donde cada producto muestra el nombre, el precio y la cantidad, y también incluye botones para incrementar o disminuir la cantidad de forma dinámica.

    \item \textbf{invoicehistoryView.dart}: Este archivo crea la clase \texttt{InvoiceHistoryView}, que básicamente maneja la visualización del historial de facturas y el total de ventas diarias. Toma una lista de \texttt{InvoiceModel}, junto con callbacks para volver a la vista principal o arrancar una nueva caja. El total se calcula solo sumando los precios multiplicados por las cantidades de todos los productos en las facturas, y lo muestra arriba, al lado del número de facturas registradas.

    Cada factura aparece en un \texttt{ExpansionTile}, donde se listan los productos con sus cantidades y subtotales, y abajo hay botones para navegar o abrir una nueva caja. Así, la clase une los datos de las facturas con una interfaz clara y práctica, haciendo que el cierre de caja sea algo simple y bien ordenado.

    \item \textbf{productsDropdown.dart}: Este archivo define la clase \texttt{ProductsDropdown}, que básicamente, es un menú desplegable que nos permite elegir productos de una lista fija. Cada opción del menú es un \texttt{DropdownMenuItem}, y el componente se encarga de recordar qué producto tenemos seleccionado en cada momento.

    En cuanto el usuario modifica la selección, el componente restablece su estado interno y avisa al controlador por medio del callback \texttt{onChanged}. Esta conexión directa hace que la elección del usuario se integre de forma inmediata con la lógica de la aplicación.
\end{itemize}

\textbf{Paso 4: Integración en la aplicación}  
Para mostrar la interfaz del punto de venta, \texttt{Home.dart} importa \texttt{PosView}. Dentro del método \texttt{build}, se agrega \texttt{PosView()} a la lista de widgets que representan las distintas secciones de la aplicación. Según el índice actual de la \texttt{NavigationBar}, se decide qué sección mostrar, de modo que al iniciar la app o al navegar desde la barra inferior, la vista del punto de venta se muestre directamente en pantalla.


\subsection{Ejercicio 12 (coins)}

La aplicación solicitada era la siguiente: \textit {“Pedir el precio de un producto y el monto que ha dado un cliente, posteriormente calcular el número mínimo de monedas a dar en el vuelto si se tienen las monedas de la siguiente denominación: 2.00, 1.00, 0.50, 0.25, 0.10.”}

Con esto en mente y siguiendo con el patrón MVC y la metodología de diseño de Atomic Design, se tiene la siguiente estructura dentro de la carpeta “coins”:

\begin{figure}[H]
    \centering
    \includegraphics[width=0.6 \textwidth, height=8cm, keepaspectratio]{ej12/cap1.png}
    \caption{Estructura de archivos en proyecto coins}
    \label{fig:ej12il1}
\end{figure}

A continuación se mostrarán cada uno de los archivos y su respectiva funcionalidad.

\textbf{vuelto model}

Esta clase define dos variables: “precio” y “monto”. Dichas variables se utilizarán para realizar el cálculo del vuelto. Para ello también se cuenta con una variable constante, que no es más que un array de 5 elementos con las denominaciones en centavos de las monedas de cambio. Esto es así para evitar tener problemas de operaciones de punto flotante.

\begin{center}
\begin{lstlisting}
const MONEDAS = [200,100,50,25,10];

class VueltoModel {

  final double precio;
  final double monto;

  VueltoModel(this.precio, this.monto);

  (List<int>, double) obtenerMonedas() {
    int cambio = ((monto - precio) * 100).round();

    var monedasVuelto = [0,0,0,0,0];
    int resto = cambio;

    int i = 0;
    while(i<5) {
      if(resto == 0){
        break;
      }

      if((resto - MONEDAS[i]) < 0) {
        i++;
        continue;
      }

      resto -= MONEDAS[i];
      monedasVuelto[i]++;
    }

    return (monedasVuelto, resto/100);
  }
}
\end{lstlisting}
\end{center}

Se tiene la función “obtenerMonedas()” que regresa dos valores, una lista con las monedas de cambio y un valor restante, porque puede haber casos donde no se alcance a dar todo el vuelo en su totalidad.

Dicha función para calcular el vuelto, primero obtiene el cambio (en centavos) que se tiene que dar. Luego se itera sobre cada denominación de moneda de mayor a menor y se le resta del monto total que se tiene que devolver con cada iteración así hasta cubrir todo el array (las 5 denominaciones). Finalmente se devuelve el array actualziado con las monedas de cambio y el valor del resto, que se lo divide para 100.

\textbf{vuelto controller}

Esta clase hace uso del “modelo” anterior. Consta de un único método “calcularCambio()” el cual recibe 2 argumentos de tipo String, que corresponden al precio y el monto que provienen de los campos de texto de la interfaz. Este método regresa dos valores, una lista con las monedas de cambio y un mensaje de error, para que pueda ser mostrado en pantalla.

\begin{center}
\begin{lstlisting}
class VueltoController{
  (List<int>, String) calcularCambio(String precioProd, String montoUsr) {...}
}
\end{lstlisting}
\end{center}

A coninuación, se realizan varias verificaciones de casos donde existan errores en la entrada de datos. Si llega a ocurrir, se devuelve una lista con ceros y el respectivo mensaje de error.

\begin{center}
\begin{lstlisting}
    // verificar que los campos estén llenados
    if(precioProd.isEmpty || montoUsr.isEmpty) {
      return (List.filled(5, 0), "Llene todos los campos para continuar");
    }

    final precio = double.tryParse(precioProd);
    final monto = double.tryParse(montoUsr);

    // verificar que sean valores double
    if(precio == null || monto == null) {
      return (List.filled(5, 0), "Ingrese un valor numérico");
    }

    // verificar que no hayan valores negativos
    if(monto < 0 || precio < 0) {
      return (List.filled(5, 0), "¡No pueden haber valores negativos!");
    }

    // verificar que el monto sea mayor que el precio
    if(monto < precio) {
      return(List.filled(5, 0), "¡El monto ingresado es menor al precio del producto!");
    }
\end{lstlisting}
\end{center}

Una vez verificados los errores de ingreso de datos, se procede a utilizar el VueltoModel para ejecutar el algoritmo para calcular el cambio en monedas llamando a su método “obtenerMonedas()”.

\begin{center}
\begin{lstlisting}
    // si no hay errores en el ingreso a datos calular el vuelto con el Model
    final vuelto = VueltoModel(precio, monto);
    final (monedas, resto) = vuelto.obtenerMonedas();
\end{lstlisting}
\end{center}

Se hace una comprobación final, puesto que dadas las denominaciones de monedas, puede haber casos donde no se pueda cubrir todo el cambio. Por ello se comprueba si hay resto, de ser el caso se devuelve la cantidad de monedas correspondiente junto con un mensaje con la cantidad de falta dar.

\begin{center}
\begin{lstlisting}
    // caso especial, si no se pudo dar todo el vuelto
    if(resto != 0) {
      return (monedas,"Faltan por dar \$${resto}" );
    }

    // si se pudo dar todo el vuelto, regresar sin mensaje de error
    return (monedas, "");
\end{lstlisting}
\end{center}

Si no hay ningún problema se regresa la cantidad de monedas con un mensaje de error en blanco.

\textbf{widgets}

\begin{itemize}
    \item \textbf{input vuelto}
\end{itemize}

Este widget es muy sencillo. Consta de un controlador (el que maneja el texto que ingresa al Input) y de un texto que se muestra como retroalimentación (label). 

Luego en términos de personalización se ajusta el widget TextField para que solo tome valores numéricos, que muestre un borde sutil y el texto de retroalimentación.

\begin{center}
\begin{lstlisting}
class InputVuelto extends StatelessWidget {
  final TextEditingController ctrl;
  final String label;

  InputVuelto({
    super.key,
    required this.ctrl,
    required this.label
  });

  @override
  Widget build(BuildContext context) {
    return TextField(
      controller: ctrl,
      keyboardType: TextInputType.number,
      decoration: InputDecoration(
        labelText: label,
        border: OutlineInputBorder(),
      ),
    );
  }
}
\end{lstlisting}
\end{center}

\begin{itemize}
    \item \textbf{coin display}
\end{itemize}

Este widget muestra un contenedor con forma circular con el valor de la denominación de la moneda y al lado muestra la cantidad respectivamente. Por tal motivo tiene dos atributos “coinValue” y “value”.

\begin{center}
\begin{lstlisting}
class CoinDisplay extends StatelessWidget {
  final String coinValue;
  final int value;

  CoinDisplay({
    super.key,
    required this.coinValue,
    required this.value
  });
\end{lstlisting}
\end{center}

Puesto que los elementos se ubican de forma horizontal, se utiliza un widget Row en primer lugar. Luego como primer elemento se tiene a un “Container” al cual se le da forma de círculo con la propiedad “shape: BoxShape.circle” y adentro del mismo muestra un texto correspondiente al valor de la denominación de la moneda.

\begin{center}
\begin{lstlisting}
  @override
  Widget build(BuildContext context) {
    return Row(
      mainAxisSize: MainAxisSize.min,
      children: [
        // Contenedor en forma de círculo
        Container(
          width: 80,
          height: 80,
          decoration: BoxDecoration(
            shape: BoxShape.circle,
            color: Colors.white,
            border: Border.all(
              color: Colors.black,
              width: 3,
            ),
          ),
          child: Center(
            child: Text(
              coinValue,
              style: TextStyle(
                fontWeight: FontWeight.bold,
                fontSize: 16,
                color: Colors.black,
              ),
              textAlign: TextAlign.center,
            ),
          ),
        ),
\end{lstlisting}
\end{center}

Luego se tiene un pequeño espaciado horizontal con \lstinline{SizedBox(width: 16),}. Finalmente se muestra un texto plano que corresponde a la cantidad que se tiene que dar de esa moneda.

\begin{center}
\begin{lstlisting}
Text(
    "$value",
    style: TextStyle(
      fontSize: 24,
      fontWeight: FontWeight.w500,
      color: value !=0 ? Colors.amber[800] : Colors.black,
    ),
),
\end{lstlisting}
\end{center}

\begin{itemize}
    \item \textbf{coin box}
\end{itemize}

Este widget utiliza el “CoinDisplay” anterior para mostrar las diferentes monedas en pantalla. Para ello se tiene un array del monto de cada moneda, el cual es pasado en su constructor.

\begin{center}
\begin{lstlisting}
import 'coin_display.dart';

class CoinBox extends StatelessWidget {
  final List<int> monedas_vuelto;

  const CoinBox({super.key, required this.monedas_vuelto});
\end{lstlisting}
\end{center}

La distribución de los elementos es de la siguiente manera. Un pequeño título al inicio, seguido de 4 widgets de CoinDisplay en una cuadrícula 4x4 y un CoinDisplay centrado en la parte del último. Por tal motivo al construir el widget, se utiliza “Column” al inicio para generar la disposición de los elementos de forma vertical.

\begin{center}
\begin{lstlisting}
  @override
  Widget build(BuildContext context) {
    return Column(
      children: [
        Center(
          child: Text(
            "Cantidad de monedas a dar",
            style: TextStyle(fontSize: 20, fontWeight: FontWeight.bold),
          ),
        ),
\end{lstlisting}
\end{center}

Para ubicar los CoinDisplay de forma contigua en horizontal, se utiliza el widget “Row”, al cual se le pasan dos widgets de CoinDisplay, cada uno se lo inicializa con el nombre de la denominación de la moneda y su valor correspondiente de cambio dado por el atributo “monedas\_vuelto” declarado al inicio.

\begin{center}
\begin{lstlisting}
SizedBox(height: 24,),

    Row(
      mainAxisAlignment: MainAxisAlignment.spaceEvenly,
      children: [
        CoinDisplay(coinValue: "\$2.00", value: monedas_vuelto[0]),
        CoinDisplay(coinValue: "\$1.00", value: monedas_vuelto[1]),
      ],
    ),

    SizedBox(height: 24,),

    Row(
      mainAxisAlignment: MainAxisAlignment.spaceEvenly,
      children: [
        CoinDisplay(coinValue: "\$0.50", value: monedas_vuelto[2]),
        CoinDisplay(coinValue: "\$0.25", value: monedas_vuelto[3]),
      ],
    ),

    SizedBox(height: 24,),

    Center(
      child: CoinDisplay(coinValue: "\$0.10", value: monedas_vuelto[4]),
    ),
\end{lstlisting}
\end{center}


\begin{itemize}
    \item \textbf{error message}
\end{itemize}

Este widget muestra un mensaje de error en pantalla cuando se necesario. Por tal motivo puede ocupar espacio en pantalla o no. Para ello primero se necesita que se le pase un texto a mostrar, el cual puede llegar a ser nulo.

\begin{center}
\begin{lstlisting}
class ErrorMessage extends StatelessWidget {
  final String? errorText;

  const ErrorMessage({
    super.key,
    this.errorText,
  });
\end{lstlisting}
\end{center}

En primera instancia, al construir el widget, se verifica si el texto suministrado es nulo o simplemente se encuentra vacío. De ser el caso, se regresa un \lstinline{SizedBox.shrink()}, para que no ocupe espacio en la pantalla.

\begin{center}
\begin{lstlisting}
  @override
  Widget build(BuildContext context) {
    if (errorText == null || errorText!.isEmpty) {
      return const SizedBox.shrink();
    }
\end{lstlisting}
\end{center}

Si hay un texto que mostrar se arma el widget como tal. Para ello, se parte desde un “Padding” que contiene un “Container” que será el que tendrá el mensaje de error y tendrá un color rojo característico, junto con bordes redondeados.

\begin{center}
\begin{lstlisting}
return Padding(
  padding: const EdgeInsets.all(16),
  child: Container(
    decoration: BoxDecoration(
      color: Color.fromRGBO(255, 0, 0, 0.1),
      borderRadius: BorderRadius.circular(12),
      border: Border.all(
        color: Color.fromRGBO(255, 0, 0, 0.1),
        width: 1.5,
      ),
    ),
\end{lstlisting}
\end{center}

El Container, tendrá un widget Padding que a su vez tendrá un widget “Expanded”. Este último es importante porque se ajustará según el espacio lo requiera. Esto porque el Expanded tendrá un widget “Text” con el mensaje de error y como este puede ser muy largo, el tamaño del contenedor se tiene que ajustar. 

\begin{center}
\begin{lstlisting}
child: Padding(
  padding: const EdgeInsets.all(16),
  child: Row(
    children: [
      Expanded(
        child: Text(
          errorText!,
          style: const TextStyle(
            color: Colors.red,
            fontSize: 18,
            fontWeight: FontWeight.w500,
          ),
          textAlign: TextAlign.start,
        ),
      ),
    ],
  ),
),
\end{lstlisting}
\end{center}

Dicho texto se encuentra estilizado, siendo de color rojo y orientado de tal manera para que pueda ser visualizado correctamente.

\textbf{vuelto view}

Esta es la vista principal en la que el usuario interactuará.Puesto que su contenido se actualiza dinámicamente, se lo define como un “StatefulWidget”, por lo que se especifica una clase para manejar su estado.

\begin{center}
\begin{lstlisting}
class VueltoPage extends StatefulWidget {
  @override
  State<VueltoPage> createState() => _VueltoPageState();
}
\end{lstlisting}
\end{center}

Dicha clase define varias variables:
\begin{itemize}
    \item Los controladores para los Inputs de precio y conto.
    
    \item El controlador del vuelto para llamar al algoritmo y manejar errores.
    
    \item Las monedas de vuelto y el mensaje de error que se mostrarán en pantalla.
\end{itemize}

\begin{center}
\begin{lstlisting}
class _VueltoPageState extends State<VueltoPage> {
  // declarar controladores
  final precioCtrl = TextEditingController();
  final montoCtrl = TextEditingController();
  final vueltoCtrl = VueltoController();

  // declarar variables
  var monedas_vuelto = [0,0,0,0,0];
  String msgError = "";
\end{lstlisting}
\end{center}

Luego se tiene un método privado “\_calcularVuelto()” que se encargará de actualizar el estado del widget; es decir, actualizará sus las variables respecto al monto de monedas y al mensaje de error.

\begin{center}
\begin{lstlisting}
void _calcularVuelto() {
    // Obtener valores usando el controllador
    final (monVuel, err) = vueltoCtrl.calcularCambio(precioCtrl.text, montoCtrl.text);

    // actualizar el estado según los valores
    setState(() {
      monedas_vuelto = monVuel;
      msgError = err;
    });
}
\end{lstlisting}
\end{center}

Luego se define el diseño a presentar, por lo que se utiliza un “Scaffold” para definir la estructura principal de una “AppBar”.

\begin{center}
\begin{lstlisting}
@override
Widget build(BuildContext context) {
  return Scaffold(
    appBar: AppBar(
      title: Text("Vuelto Mínimo"),
      backgroundColor: Colors.blue[900],
      foregroundColor: Colors.white,
      centerTitle: true,
    ),
\end{lstlisting}
\end{center}

Finalmente se hace uso de los widgets definidos anteriormente para generar la pantalla principal dentro del \lstinline{body} del Scaffold.

\begin{center}
\begin{lstlisting}
child:  Column(
  crossAxisAlignment: CrossAxisAlignment.stretch,
  children: [
    Text(
      "Ingrese el precio del producto y el monto a pagar",
      style: TextStyle(fontSize: 16, fontWeight: FontWeight.bold),
    ),
    SizedBox(height: 12,),
    InputVuelto(ctrl: precioCtrl, label: "precio (2 decimales)"),
    SizedBox(height: 12,),
    InputVuelto(ctrl: montoCtrl, label: "monto (2 decimales)"),
    SizedBox(height: 12,),
    ElevatedButton(
        onPressed: _calcularVuelto,
        child: Text("Calcular")
    ),
    ErrorMessage(errorText: msgError),
    SizedBox(height: 18,),
    CoinBox(monedas_vuelto: monedas_vuelto),
  ],
),
\end{lstlisting}
\end{center}

\subsection{Ejercicio 1 (years)}
Contenido desde un archivo a parte

\subsection{Ejercicio 14 (perfect)}
\textbf{Paso 1: Estructuración según el patrón MVC}  
Para estructurar el módulo de números perfectos, se siguió nuevamente el patrón Modelo-Vista-Controlador (MVC). Se crearon archivos separados para el modelo (PerfectNumber), el controlador (PerfectController) y la vista (PerfectPage), lo que mantiene la lógica de cálculo, la gestión de datos y la interfaz de usuario bien diferenciadas y fáciles de mantener dentro del proyecto.
Se crearon tres carpetas principales dentro del proyecto:  
\begin{itemize}
    \item model
    \item view
    \item controller
\end{itemize}

\textbf{Paso 2: Creación de archivos .dart para cada componente}  
Dentro de cada carpeta se añadieron los archivos correspondientes en el siguiente orden de desarrollo:  

\begin{itemize}
    \item \textbf{Model:} 
        \begin{itemize}
            \item perfectModel.dart
        \end{itemize}
    \item \textbf{Controller:} 
        \begin{itemize}
            \item perfectController.dart
        \end{itemize}
    \item \textbf{View:} 
        \begin{itemize}
            \item perfectView.dart
        \end{itemize}
\end{itemize}

\textbf{Paso 3: Desarrollo de los componentes (modelo $\rightarrow$ controlador $\rightarrow$ vista)}  

\textbf{Modelos:}  

\begin{itemize}
    \item \textbf{perfectModel.dart}: La clase \texttt{PerfectNumber} básicamente se dedica a revisar si un número es perfecto, que se base en la programación donde si sumas todos sus divisores (excluyendo el propio número) el resultado es exactamente el mismo número. Al crear una instancia con un número, la clase automáticamente calcula sus divisores y determina si cumple o no con esta condición.

    Dentro de la clase, la lógica se apoya principalmente en dos métodos: \texttt{\_calculateDivisors}, que identifica y ordena todos los divisores menores que el número, y \texttt{\_checkIfPerfect}, que suma esos divisores para confirmar si el número es perfecto. Además, incluye métodos adicionales que hacen posible alcanzar la suma de los divisores y presentarlos como texto, lo que sirve mucho tanto para mostrarlos al usuario como para aprovechar los datos en otras partes de la app.
\end{itemize}

\textbf{Controladores:}  

\begin{itemize}
    \item \textbf{perfectController.dart}: La clase \texttt{PerfectController} gestiona toda la lógica detrás de la verificación de números perfectos. Para ello, importa la clase \texttt{PerfectNumber} y proporciona el método \texttt{checkNumber}, que trabaja con un número, si es mayor que cero, crea una instancia de \texttt{PerfectNumber} para comprobar sus divisores y comprobar si verdaderamente es un número perfecto. De esta manera, el controlador unifica tanto la validación como la creación del modelo, evitando el manejo de información errónea y manteniendo la comunicación clara y ordenada entre la lógica del programa y la interfaz de usuario.
\end{itemize}

\textbf{Vista:}  

\begin{itemize}
    \item \textbf{perfectView.dart}: En este archivo se define la interfaz en Flutter que permite comprobar si un número es perfecto. Para hacerlo, se utiliza un \texttt{NumberField}, donde el usuario ingresa el número, un \texttt{PrimaryButton} que ejecuta la verificación y el componente \texttt{PerfectCard}, que actúa como el enlace entre lo que el usuario introduce y la lógica del \texttt{PerfectController}, encargado de calcular los divisores y determinar si el número cumple la condición.
    
    La página principal, \texttt{PerfectPage}, está construida sobre un \texttt{Scaffold} que integra esta tarjeta de verificación. Lo interesante es que los resultados se actualizan al instante: muestra si el número es perfecto y también la suma de sus divisores. En esencia, esta pantalla conecta de forma directa la interfaz con la lógica del controlador, logrando que la interacción sea fluida y los cálculos se reflejen en tiempo real.
\end{itemize}

\textbf{Paso 4: Integración en la aplicación}  
Para integrar la funcionalidad de números perfectos, lo que hicimos fue importar \texttt{PerfectPage} en el \texttt{Home.dart} y agregarla como una opción más en la lista de secciones.

Así, cuando el usuario elige esta opción en la barra de navegación, la página carga automáticamente y le permite ingresar un número para verificar si es perfecto, mostrando también sus divisores. De esta forma, queda totalmente integrado en la navegación principal de la aplicación.

\subsection{Ejercicio 15 (diet)}
Desarrollo del ejercicio 15

%===========================================================
%===========================================================
% --- Conclusiones y Recomendaciones ---
%===========================================================
%===========================================================

\section{Conclusiones y Recomendaciones}
\subsection{Conclusiones}

\begin{itemize}
    \item Se logró desarrollar una aplicación móvil en Flutter la cual cuanta con un sistema de navegación que permite navegar entre cinco ejercicios diferentes. Gracias a esto se pudieron usar bien varias de las herramientas del framework y se aplicaron principios de diseño de interfaces móviles, usando widgets fundamentales para crear pantallas interactivas capaces de capturar y mostrar los datos de forma fácil de entender y bien organizada
    \item Se dejó el código bien ordenado y fácil de entender, separando en carpetas distintas lo que corresponde al diseño y lo que corresponde a la lógica de programación. De esta manera, se siguieron buenas prácticas en la fase de desarrollo lo que permitió tener beneficios como facilidad de modificación de código.
    \item Seguir los principios de Atomic Design, permitió crear interfaces de manera modular, lo que las hizo más fáciles de reutilizar. Aplicar estos principios permitió organizar de manera mas clara los componentes en diferentes niveles como: átomos, moléculas y organismos, Esto hace que el desarrollo sea más claro y que mantener la aplicación resulte más sencillo y practico.
    \item Al trabajar con el patrón MVC (Modelo-Vista-Controlador), se consiguió mantener la lógica de negocio separada de toda la parte visual de la aplicación multifuncional, esto hizo que el código fuera más claro y organizado, también hace que sea mas mantenible y tenga una base mas solida en caso de que se requieran futuras expansiones.
\end{itemize}


\subsection{Recomendaciones}

\begin{itemize}
    \item Se recomienda continuar aplicando en futuros proyectos la metodología de Atomic Design, porque esta ayuda a crear interfaces modulares, escalables y fáciles de reutilizar. Seguir con esta forma de trabajar hace que agregar nuevos componentes o ampliar la aplicación sea mucho más fácil y sencillo, sin tocar la estructura de lo que ya está funcionando.

    \item Resulta clave mantener la utilización del patrón MVC(Modelo-Vista-Controlador) siempre que se pueda aplicar, ya que los beneficios que brinda este patrón como: separar la lógica de negocio de la presentación. No solo hace que el código sea más claro y fácil de mantener, sino que también simplifica y hace mas sencilla la tarea de incorporar nuevas funciones, además este patrón puede facilitar proyectos que puedan considerarse complejos.
    
    \item Se recomienda documentar y estandarizar los componentes y Widgets creados en las aplicaciones, enfocándose en hacerlos reutilizables dentro del mismo proyecto o incluso en otros proyectos. Esto facilita obtener el máximo beneficio de  la modularidad que ofrece Atomic Design y la clara separación de responsabilidades que da el patrón MVC, garantizando consistencia, un mantenimiento más sencillo y la posibilidad de escalar la aplicación sin tener muchos problemas.
\end{itemize}


%===========================================================
%===========================================================
% --- Referencias Bibliográficas ---
%===========================================================
%===========================================================

\section{Referencias Bibliográficas}
\printbibliography[heading=none]

%===========================================================
%===========================================================
% --- Anexos ---
%===========================================================
%===========================================================

\section{Anexos}
\textbf{Anexo 1. Pantallas en ejecución}

\begin{figure}[H]
    \centering
    \includegraphics[width=0.9 \textwidth, height=16cm, keepaspectratio]{anexos/anx1_1.png}
\end{figure}

\begin{figure}[H]
    \centering
    \includegraphics[width=0.9 \textwidth, height=16cm, keepaspectratio]{anexos/anx1_2.png}
\end{figure}

\end{document}
